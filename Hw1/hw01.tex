\documentclass[letterpaper,11pt]{article}

\usepackage{fullpage}
\usepackage{amsmath}
\usepackage{amsthm}
\usepackage{hyperref}

\newtheorem{theorem}{Theorem}[section]

\begin{document}

\noindent{Adam Pettway}

\noindent{Colgate University} 

\noindent{\bf Problem Set 1 --- Invariants and Induction} 

\begin{enumerate}
\item Problem 2.1-3 \\
The variation of the sequencial search detailed in the problem does have different effciencies. For the best case effciency the classical sequencial search is  1. In this case, the search key is found in the first element so the algorithm terminates after one loop. However, in the variation all elements need to be searched since a list of key locations need to be returned. And the key may be in the last element. The worst case is the same in both versions, since the key is found in the last element the entire array needs to be searched which makes the effciency n. Finally, for the average effciency they differ. The average effciency for the variation is n since all elements are accessed.
\item Problem 2-2 in CLRS, 3rd edition. 
\begin{enumerate}
\item 
Prove that $\frac{n(n+1)}{2} \in O(n^3)$.\\

\begin{proof}
We show that $\exists c>0,n_0>0 \text{ s.t. } 0 \leq \frac{n(n+1)}{2} \leq cn^3$, $\forall n \geq n_0$.
\begin{align*}
\frac{n(n+1)}{2}&\leq \frac{n^2+n}{2}&n\geq 1\\
            &\leq n^3 + n&n\geq 1\\
            &\leq 2n^3&n\geq 1
\end{align*}
Furthermore, we have that $\frac{n(n+1)}{2} \geq 0$ for $n\geq 1$.
Therefore for $c=2$, and all $n\geq n_0=1$, we have that $0\leq \frac{n(n+1)}{2}  \leq cn^3$ and by definition $\frac{n(n+1)}{2}  \in O(n^3)$.
\end{proof}

\item
Prove that $\frac{n(n+1)}{2} \in O(n^2)$.\\

\begin{proof}
We show that $\exists c>0,n_0>0 \text{ s.t. } 0 \leq \frac{n(n+1)}{2} \leq cn^2$, $\forall n \geq n_0$.
\begin{align*}
\frac{n(n+1)}{2}&\leq \frac{n^2+n}{2}&n\geq 1\\
            &\leq n^2 + n&n\geq 1\\
            &\leq 2n^2&n\geq 1
\end{align*}
Furthermore, we have that $\frac{n(n+1)}{2} \geq 0$ for $n\geq 1$.
Therefore for $c=2$, and all $n\geq n_0=1$, we have that $0\leq \frac{n(n+1)}{2}  \leq cn^3$ and by definition $\frac{n(n+1)}{2}  \in O(n^3)$.
\end{proof}

\item
Prove that $\frac{n(n+1)}{2} \in \Theta(n^2)$.\\

We show that $\frac{n(n+1)}{2} \in O(n^2)$ and that $\frac{n(n+1)}{2} \in \Omega(n^2)$. By Theorem 3.1, this shows that $\frac{n(n+1)}{2} \in \Theta(n^2)$.

First we show that $\frac{n(n+1)}{2} \in O(n^2)$. Which is done in part $(b)$. \\

Next, we show that $\frac{n(n+1)}{2} \in \Omega(n^2)$.
\begin{proof}
We show that $\exists c>0,n_0>0 \text{ s.t. } 0 \leq cn^2 \leq \frac{n(n+1)}{2}$, $\forall n \geq n_0$.
\begin{align*}
\frac{n(n+1)}{2} &\geq  \frac{n+1}{2} & \forall n\\
            &\geq n^2 + n^2&n\geq 1\\
            &=    2n^2&n\geq 1\\
            &\geq 0 &n\geq 1
\end{align*}
Furthermore, we have that $n^2 + n + 5 \geq 0$ for $n\geq 1$.
Therefore for $c=2$, and all $n\geq n_0=1$, we have that $0\leq cn^2 \leq n^2 + n + 5$ and by definition $n^2 + n + 5 = \Omega(n^2)$.
\end{proof}
\item a durian
\end{enumerate}
\item Prove by induction that for every non-negative integer $n$
\[\sum_{k=0}^{n} k^2 = \frac{n(n+1)(2n+1)}{6}.\]
\item Prove that given an unlimited supply of 6-cent coins, 10-cent coins, and 15-cent coins, one can make any amount of change larger than 29 cents.
\end{enumerate}

\end{document}
