\documentclass[letterpaper,11pt]{article}
\usepackage{graphicx}
\usepackage{fullpage}
\usepackage{amsmath}
\usepackage{amsthm}
\usepackage{hyperref}
\usepackage{algorithm}
\usepackage[noend]{algpseudocode}
\newtheorem{theorem}{Theorem}[section]

\begin{document}
\noindent{COSC 302: Analysis of Algorithms --- Spring 2018}

\noindent{\bf Problem Set 7 --- Minimum Spanning Trees and Single Source Shortest Paths} 
\noindent {\bf Due by 4:30pm Friday, March 30, 2018 as a single pdf via Moodle (either generated via \LaTeX{}, or concatenated photos of your work). Late assignments are not accepted.} 

This is an \emph{individual} assignment: collaboration (such as discussing problems and brainstorming ideas for solving them) on this assignment is highly encouraged, but the work you submit must be your own. Give information only as a tutor would: ask questions so that your classmate is able to figure out the answer for themselves. It is unacceptable to share any artifacts, such as code and/or write-ups for this assignment. If you work with someone in close collaboration, you must mention your collaborator on your assignment.


\begin{enumerate}
\item Problem 23.1-1 from CLRS.
Let $G = (V,E)$ be a connected undirected graph. Let $(u,v)$ be a minimum weight edge. Let there be $T$ which is some subset of $E$ that does not contain a vertex $u$. Since $(u,v)$ is a minimum weight edge, it is a safe edge. And a minimum spanning tree must contain all vertices including $v$ the edge $(u,v)$ will be added.
\item Problem 23.2-8 from CLRS.
With the valid input where a tree that has lower weight exists.
  \includegraphics{figure1}
\item Problem 24.1-6 from CLRS.
\begin{algorithm}
\begin{algorithmic}[1]
\Function{Negative-Cycle}{$G$}
\State $Q = \emptyset$
\ForAll{$(u,v) \in E$}
\State $E = E - (u,v)$
\If{\Call{Bellman-Ford}{$G$} == true}
\State Q = Q + $u_1$
\State $u = u_1$
\State $E = E + (u,v)$
\ForAll{$x \in adj[v]$}
\State $E = E - (v,x)$
\If{\Call{Bellman-Ford}{$G$} == true}
\State $Q = Q + u$
\State $E = E + (v,x)$
\State $u = v$
\State $v = x$
\If{$x = u_1$}
\State \Return Q
\EndIf
\EndIf
\EndFor
\EndIf
\EndFor
\State \Return Q
\EndFunction
\end{algorithmic}
\end{algorithm}
\\Invariant: The negative weight cycle is intact right before the start of every loop.
\\Initialization: At the start of the loop, no edges have been removed nor recorded.
\\Maintanence: At the start of the loop we first check if Bellman-Ford function returns true. If it returns false we can remove an edge from the graph without changing the cycle and thus passes the invariant. Then when Bellman-Ford returns true, it means that the edge $u,v$ was a part of the cycle, thus we add back the edge and using the adjacency list do a breadth first search to find the remaining parts of the cycle. If we find another edge $(v,x)$ when removed causes the BF return true then we change $u$ to equal $v$ and $v$ to equal the current $x$ so that we search it's neighbors for another edge in the cycle.
\\Termination: The function terminates if there was no cycle found. It also terminates when $x = u_1$ with $u_1$ being the first vertex found to the part of the cycle.
\item Problem 24.3-4 from CLRS.
\begin{enumerate}
\item We first check if there is one vertex $s$ which $\pi[s] = NIL$ and $dis[s] = 0$.
\item We then check if the information given is a tree on the graph given. We first check if $dis.length = \pi.length -1 $. We then do a BFS and check if it visits all = and $\pi[v] \neq \infty$.Then, we check if for each pair $(u,v)$, $dis[u] + w(u,v) \geq dis[v]$ 
\item We check for each $v \in \pi - s$, if $dis[\pi[v]] + w(\pi[v],v) = dis[v]$
\end{enumerate}
\end{enumerate}
\end{document}
