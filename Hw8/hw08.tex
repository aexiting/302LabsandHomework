\documentclass[letterpaper,11pt]{article}
\usepackage{amsfonts}
\usepackage{fullpage}
\usepackage{graphicx}

\usepackage[noend]{algpseudocode}
\usepackage{algorithm}

\newtheorem{theorem}{Theorem}[section]

\begin{document}
The vehicle Routing Problem or VRP is a problem is finding the optimal route for a number of vehicles to traverse the set of stops and roads that connect these stops. Each stop may or may not have a customer which requires a certain number of each resource. And usually, each vehicle can only carry a finite amount of materials. Some common goals of optimization is minimizing the driving time for each vehicle or using the least amount of vehicles to reach all customers under a time constraint. These goals can be considered constraints for a valid answer. Naturally, with the more constraints, the problem becomes more difficult to optimize, since one solution which improves one component often does so at the other’s expense.  This problem has countless applications in distribution.\\

The graph problem defined by Gilbert Laporte, is as follows: Let $G=(V,A)$ be a graph where $V={1,2...n}$ is a set of vertices representing cities with the depot located at vertex $1$, and $A$ is the set of arcs. With every arc $(i,j) i \neq j$ is associated a non-negative distance matrix $C = (c_{ij})$. Note that $c_{ij}$ can also be interpreted as a travel cost or a travel time. When $C$ is symmetrical, we can replace $A$ with $E$ which is a set of undirected edges.  In addition, assume there are $m$ vehicles available at the depot, where they are all located at the start of the algorithm. (Fin Gilbert Laporte)

 
\end{document}
