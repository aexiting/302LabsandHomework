\documentclass[letterpaper,11pt]{article}

\usepackage{fullpage}
\usepackage{amsmath}
\usepackage{amsthm}
\usepackage{hyperref}

\newtheorem{theorem}{Theorem}[section]

\begin{document}
\noindent{COSC 302: Analysis of Algorithms --- Spring 2018}

\noindent{Adam Pettway}

\noindent{Colgate University} \\

\noindent{\bf Problem Set 3 --- Solving Recurrences and Divide and Conquer I} \\



\begin{enumerate}
\item Problem 4-1 from CLRS 3rd edition.
\begin{enumerate}
\item[a]The third case of the master method applies since  $f(n)$ dominates the running time therefore it's $\theta(n^4)$.

\item[b] The third case of the master method applies since $f(n)$ domintates the running time, therefore it's $\theta(n)$.
\item[c] The second case of the master method applies since the running time is evenly distrubuted, therefore it is $\theta (n^2log(n))$.
\item[d] The third case of the master method applies since $f(n)$ dominates the running time, therefore it's $\theta(n^2)$.
\item[e] 
The first case of the master method applies since  $n^log_2(7)$ dominates, therefore it's $\theta(n^{2.807}) $.
\item[f] The first case of the master method applies since the running time is evenly distrubuted, therefore it is $\theta(\sqrt[]{n}log(n))$.
\item[g] The master method does not apply since there is no $b$ value. Therefore we shall solve the recurrence by subsitution. First we start with the recurrence $T(n) = T(n-2) + n^2$. And we then subsitute $n-1$ into the recurrence to get $T(n-1) = T(n-3) + T(n-2)^2 + n^2$. It's clear that there will be $n$ number of recursive calls therefore, the running time is $\theta (n^2)$.

\end{enumerate}
\item Using one of the methods discussed in lecture, give a tight asymptotic bound for the recurrence $T(n) = 8T(n/2) + n^3 \log n$. \\
The third case of the master method applies since $f(n)$ dominates the running time since $n^3 = O(n^3\log(n))$ . $f(n)$ also passes the regularity tests since $8\frac{n}{2}^3log(\frac{n}{2}) \leq cn^3\log(n)$ when $c=8$.
\item Problem 4.3-9 from CLRS 3rd edition.

\item \emph{Revisiting the pinePhone}. You are still hard at work testing the quality of pinePhones for Pineapple. 

\begin{enumerate}
\item With $3$ pinePhones, one can divide the ladder by $\sqrt[3]{n}$ parts. Then start at the highest rung of each part $(c*\sqrt[3]{n}$ with $1 \leq c \leq n/\sqrt[3]{n})$. If one of these highest rungs break then check the lower ladders by $1$. Which would lead to a maximum of $3$  phones broken.
\item YOne can find the highest safe rung with $4$ pinePhones with $\Theta(\sqrt[4]{n})$ pinePhone drops.
With $3$ pinePhones, one can divide the ladder by $\sqrt[4]{n}$ parts. Then start at the highest rung of each part $(c*\sqrt[4]{n}$ with $1 \leq c \leq n/\sqrt[4]{n})$. If one of these highest rungs break then check the lower ladders by $1$. Which would lead to a maximum of $4$  phones broken.
\item 
$\begin{cases} 
       n &  k = 1\\
      T(\sqrt[k]{n})+\theta(1) & k \geq 2\\
 \end{cases}$
\end{enumerate}
\end{enumerate}

\end{document}
