\documentclass[letterpaper,11pt]{article}

\usepackage{fullpage}
\usepackage{amsmath}
\usepackage{amsthm}
\usepackage{hyperref}
\usepackage[noend]{algpseudocode}
\usepackage{algorithm}
\usepackage{fullpage}
\usepackage{amsmath}
\usepackage{amsthm}
\usepackage{hyperref}

\newtheorem{theorem}{Theorem}[section]

\begin{document}
\noindent{COSC 302: Analysis of Algorithms --- Spring 2018}

\noindent{Adam Pettway}

\noindent{Colgate University} 

\noindent{\bf Problem Set 2 --- Order of Growth and Asymptotic Analysis} 

\begin{enumerate}
\item Describe a $\Theta(n\log n)$-time algorithm that, given an array $A$ of $n$ integers and another integer $x$, determines whether or not there exist two elements in $A$ whose sum is exactly $x$. (\emph{Hint: do not use divide-and-conquer.})

\item Problem 3.1-1 in CLRS. \\
We shall prove that that $max(f(n),g(n)) = \Theta(f(n)+g(n))$. 

Since $f(n)$ and $g(n)$ are nonnegative functions $f(n) \leq f(n)+g(n)$ and  $g(n) \leq f(n)+g(n)$. Therefore, $max(f(n),g(n)) = O(f(n)+g(n))$.

Note that $f(n) + g(n) \leq 2*max(f(n),g(n))$ because $f(n) \leq max(f(n),g(n))$ and
$g(n) \leq max(f(n),g(n))$ by definition.

Also, the definition of $\Omega$ if there exists a positive constant $k$ which $0 \leq cg(n) \leq f(n) \forall n \geq n_0$. In this case, $n_0 = 0 and k = 2$.

Since we proved that $max(f(n),g(n)) = \Omega(f(n)+g(n))$ and $max(f(n),g(n)) = O(f(n)+g(n))$ then by definition $max(f(n),g(n)) = \Theta(f(n)+g(n))$.

\item Problem 3-2 (Big Oh, Big Omega, and Big Theta only) in CLRS.


\begin{table}[h!]
\centering
 \begin{tabular}{|c c c c c|} 
 \hline
 $A$ & $B$ & $O$ & $\Omega$ & $\Theta$\\ [0.5ex] 
 \hline\hline
 $lg^{k}n$ &  $n^\epsilon$ & No & Yes & No\\ 
 $n^k$ & $c^n$ & Yes & No &No\\
 $\sqrt[]{n}$ & $n^{\sin{n}}$ & No & No &No\\
  $2^n$&  $2^{n/2}$ & Yes & Yes &Yes\\
$n^{lg{c}}$ & $c^{lg{n}}$ & Yes & No &No\\ 
$lg(n!)$ & $lg(n^n)$  & Yes & No &No\\[1ex] 
 \hline
 \end{tabular}
\end{table}


\item Problem 3-4 (parts c,d,e,g) in CLRS.
\begin{enumerate}
\item[c]
$f(n) = O(g(n)) \implies lg(f(n)) = O(g(n))$. We will prove this directly.
By defition of upper bound: $f(n) \leq g(n)$ and by definition of log  $lg(f(n)) \leq lg(g(n))$ which means that  $lg(f(n)) = O(lg(g(n)))$ 
\item[d]
Suppose that $f(n) = 2n$ and $g(n) = lg n$. Then by definition, $f(n) = O(g(n)$ when $c = 2$. However, $2^{2n} \neq O(2^{n})$ since there is no $c$ that works so that so that $f(n) \leq g(n)$ for all $n$.  Therefore the statement is not always true.
\item[e]
Suppose that $f(n) = \frac{1}{n}$ thus $(f(n))^2 = \frac{1}{n^2}$. So the statement  $f(n) = O((f(n))^2)$ is a contradiction since there exists no $c$ that works so that for all $n$ $f(n) \leq (f(n))^2$. Therefore the statement is not true.
\item[g]
Suppose that $f(n) = 2^n$  so we must show that $2^n \leq  c*2^{\frac{n}{2}}$ If we simplify this we get $n \leq c*\frac{n}{2}$. But there exists no $c > 1$ where $n \leq c*\frac{n}{2}$ for all $n$. Therefore the statement is not true.
\end{enumerate}
\end{enumerate}

\end{document}
