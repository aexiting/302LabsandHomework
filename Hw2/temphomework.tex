\documentclass[letterpaper,11pt]{article}

\usepackage{fullpage}
\usepackage{amsmath}
\usepackage{amsthm}
\usepackage{hyperref}
\usepackage[noend]{algpseudocode}
\usepackage{algorithm}
\usepackage{fullpage}
\usepackage{amsmath}
\usepackage{amsthm}
\usepackage{hyperref}
\usepackage{tikz-qtree}
\newtheorem{theorem}{Theorem}[section]

\begin{document}
\noindent{COSC 302: Analysis of Algorithms --- Spring 2018}

\noindent{Adam Pettway}

\noindent{Colgate University} 

\noindent{\bf Problem Set 2 --- Order of Growth and Asymptotic Analysis} 
\begin{algorithm}[!h]\begin{algorithm}[!h]
\textsc{Hobby-Majority}($A$)
\begin{algorithmic}
\If{ $A.length = 1$}
\State \Return $A[1]$
\EndIf
\State $x \gets$ Hobby-Majority($[1...n/2]$)
\State $y \gets$ Hobby-Majority($A[(n/2)+1...n]$)
\If{SameHobby($x,y$)}
\State \Return x
\Else
    \If{x = NULL}
    \State \Return y
    \EndIf
    \If{y = NULL}
    \State \Return x
    \EndIf
\EndIf \\
\Return NULL
\end{algorithmic}
\end{algorithm}


\textsc{Hobby-Majority}($A$)
\begin{algorithmic}
\If{ $A.length = 1$}
\State \Return $A[1]$
\EndIf
\State $x \gets$ Hobby-Majority($[1...n/2]$)
\State $y \gets$ Hobby-Majority($A[(n/2)+1...n]$)
\If{SameHobby($x,y$)}
\State \Return x
\Else
    \If{x = NULL}
    \State \Return y
    \EndIf
    \If{y = NULL}
    \State \Return x
    \EndIf
\EndIf \\
\Return NULL
\end{algorithmic}
\end{algorithm}


\begin{enumerate}
\item
\textsc{Linear-Median}($A$, $x$)
\begin{algorithmic}[1]
\State $median \gets Median(A)$
\State $w_{<M} \gets 0$
\State $Array_{<M} \gets \emptyset$
\State $Array_{>M} \gets \emptyset$
\If {$A.length = 1$}
\State \Return $1$
\EndIf
\For{$j = 1$ \textbf{to} $A$.length}
\If{$A[j] \leq median$}
\State $Array_{<M} = Array_{<M} + A[j]$
\State $w_{<M} = w_{<M} + w(A[i])$  
\Else 
\State $Array_{>M} = Array_{>M} + A[j]$
\EndIf
\EndFor
\If{$x + w_{<M} > \frac{1}{2}$}
\State Linear-Median($Array_{<M}$,$x$)
\Else
\State Linear-Median($Array_{>M}$,$w_{<M}$)
\EndIf
\end{algorithmic}
\item Invariant: The median is also in the array $A$ and $x$ is the total weight of all elements than the median the previous call's sub-array.\\

Initialization: For the first call $A$ is the total array and the weight of elements less than the median is $0$.\\

Maintenance: Suppose that $x + w_{<M} > \frac{1}{2}$ since the median is in A, it must be in either $Array_{<M}$ or $Array_{>M}$ since we iterated through $A$ and placed the median in one of those arrays. Since the total weight of all elements less than the any element is greater than $\frac{1}{2}$ then it must not be in array $Array_{>M}$ and it will be in $Array_{<M}$. Otherwise it will be in $Array_{>M}$ since we do not remove any elements from the array. \\

Termination: The program terminates when the array is length of 1 since the size of $A$ decreases for every recurse since neither arrays can contain the same elements. The recurrence is $T(n) = T(n/2) + \Theta(n)$ which equals $\Theta(n)$.
\item The trees go left to right starting with the upper right one.

\item Question 3.
\begin{enumerate}
\item[a] There is a $\frac{1}{n}$ chance of the number being the maximum. 
\end{enumerate}
\end{enumerate}
\tikzset{every tree node/.style={minimum width=2em,draw,circle},
         blank/.style={draw=none},
         edge from parent/.style=
         {draw,edge from parent path={(\tikzparentnode) -- (\tikzchildnode)}},
         level distance=1.5cm}
\begin{tikzpicture}

\Tree
[.10     
    [.9
     [.5
		[.2
             \edge[blank]; 
             \edge[blank]; \node[blank]{};
         ]
         [.1
             \edge[blank]; 
             \edge[blank]; \node[blank]{};
         ]
         ]     
     [.3
          [.0
             \edge[blank]; 
             \edge[blank]; \node[blank]{};
         ]
         [
             \edge[blank]; 
             \edge[blank]; \node[blank]{};
         ]
         ]
    ]
    [.6 
	[.8
	 \edge[blank]; 
             \edge[blank]; \node[blank]{};]     
     [.7
             \edge[blank]; 
             \edge[blank]; \node[blank]{};
         ]
    ]
]
\end{tikzpicture}
\tikzset{every tree node/.style={minimum width=2em,draw,circle},
         blank/.style={draw=none},
         edge from parent/.style=
         {draw,edge from parent path={(\tikzparentnode) -- (\tikzchildnode)}},
         level distance=1.5cm}
\begin{tikzpicture}

\Tree
[.10     
    [.9
     [.5
		[.2
             \edge[blank]; 
             \edge[blank]; \node[blank]{};
         ]
         [.1
             \edge[blank]; 
             \edge[blank]; \node[blank]{};
         ]
         ]     
     [.3
          [.0
             \edge[blank]; 
             \edge[blank]; \node[blank]{};
         ]
         [
             \edge[blank]; 
             \edge[blank]; \node[blank]{};
         ]
         ]
    ]
    [.8 
	[.6
	 \edge[blank]; 
             \edge[blank]; \node[blank]{};]     
     [.7
             \edge[blank]; 
             \edge[blank]; \node[blank]{};
         ]
    ]
]
\end{tikzpicture}
\tikzset{every tree node/.style={minimum width=2em,draw,circle},
         blank/.style={draw=none},
         edge from parent/.style=
         {draw,edge from parent path={(\tikzparentnode) -- (\tikzchildnode)}},
         level distance=1.5cm}
\begin{tikzpicture}

\Tree
[.9     
    [.5
     [.2
		[.1
             \edge[blank]; 
             \edge[blank]; \node[blank]{};
         ]
         [.0
             \edge[blank]; 
             \edge[blank]; \node[blank]{};
         ]
         ]     
     [.3
      \edge[blank]; 
             \edge[blank]; \node[blank]{};
      ]
    ]
    [.8 
	[.7
	 \edge[blank]; 
             \edge[blank]; \node[blank]{};]     
     [.6
             \edge[blank]; 
             \edge[blank]; \node[blank]{};
         ]
    ]
]
\end{tikzpicture}

\tikzset{every tree node/.style={minimum width=2em,draw,circle},
         blank/.style={draw=none},
         edge from parent/.style=
         {draw,edge from parent path={(\tikzparentnode) -- (\tikzchildnode)}},
         level distance=1.5cm}
\begin{tikzpicture}

\Tree
[.9     
    [.5
     [.4
		[.1
             \edge[blank]; 
             \edge[blank]; \node[blank]{};
         ]
         [.2
             \edge[blank]; 
             \edge[blank]; \node[blank]{};
         ]
         ]     
     [.3
      \edge[blank]; 
             \edge[blank]; \node[blank]{};
      ]
    ]
    [.8 
	[.7
	 \edge[blank]; 
             \edge[blank]; \node[blank]{};]     
     [.6
             \edge[blank]; 
             \edge[blank]; \node[blank]{};
         ]
    ]
]
\end{tikzpicture}

\end{document}
